% Metódy inžinierskej práce

\documentclass[10pt,twoside,english,a4paper]{article}

\usepackage[english]{babel}
%\usepackage[T1]{fontenc}
\usepackage[IL2]{fontenc} % lepšia sadzba písmena Ľ než v T1
\usepackage[utf8]{inputenc}
\usepackage{graphicx}
\usepackage{subcaption}
\usepackage{url} % príkaz \url na formátovanie URL
\usepackage{hyperref} % odkazy v texte budú aktívne (pri niektorých triedach dokumentov spôsobuje posun textu)
\usepackage{float}
\usepackage{wrapfig}
\usepackage{mathtools}

\usepackage{cite}
%\usepackage{times}

\usepackage{geometry}
\geometry{top=1in, bottom=1in, left=1.5in, right=1.5in}

\pagestyle{headings}

\title{Recommendation systems for personalized advertising in digital marketing} % meno a priezvisko vyučujúceho na cvičeniach


\author{Vsevolod Salik\\[2pt]
	{\small Slovak University of Technology in Bratislava }\\
	{\small Faculty of Informatics and Information Technologies }\\
	{\small \texttt{xsalik@stuba.sk}}
	}

\date{\small 26. september 2024}



\begin{document}


\maketitle
%\thanks{Semestral project in the subject Methods of engineering work, academic year 2024/25, supervised by Pavol Batalik}
%insert thanks part in the title after i figure out how it works



\begin{abstract}
\ldots
\end{abstract}



\section{Introduction}
We live in a digital era, where technology plays an essential role in enhancing quality of life. Internet , one of the most transformative inventions, is the the largest repository of information ever created which can be used from almost every part of the world. Humanity gained an access to a such powerful tool which is beyond human capabilities, making it difficult to retain all the information encountered daily. That is the part where recommendation systems step in.

\medskip Recommendation systems provide information based on prior user interactions, tailored to match individual interests. They can be applied across various types of data, such as music, videos, articles, shopping, and services. This article focuses on personalized advertising in the field of digital marketing.

\medskip Competition on a marketplace grows daily. Targeting the appropriate audience and effectively selling products has become more complex comparing to previous years and that is why confronting a customer to make a purchase by simply showing them the advertisement is no longer effective enough. 

\medskip The main focus of an advertising industry is not to send the ads to everyone but to find appropriate customers, who can be potentially interested in your product, afterward, create personalized advertisements, which will target them, to satisfy both customer needs and business financial objectives. Recommendation systems make it possible to identify relevant customers, tailoring content to each individual, recognizing preferences, and increasing their engagement. These are the primary functions of recommendation systems in digital marketing.

\medskip This article will briefly explore recommendation systems templates, describe principles of their work, examine the role of artificial intelligence, and present custom recommender engine implementations. The goal is to optimize the use of recommendation systems to create personalized advertising that increase sales efficiency.

\section{Recommendation systems}

As shown on a figure \ref{fig:fig1}, model of an RS consists of user, item resource and recommendation algorithm \cite{8506344}. 
Among these components, the recommendation algorithm is the most important part of RS\cite{ren2012research}\cite{zhang2015research}
The user model consists of computed preferences derived from personal data, including search history, purchase history, saved webpages, and so on.

Following this an item within the user’s field of interest is selected, and a recommendation is generated for presentation to the user.

The performance of a recommendation system is not covered in this section ,as it is directly related to the recommendation algorithm, which will be described in subsequent sections.

\begin{figure}[H]
    \includegraphics[width=1\textwidth]{./diagrams/recommender_system_model.png}
    \caption{Model of recommender systems. Reproduced from \cite{8506344}}
    \label{fig:fig1}
\end{figure}


\subsection{Types of recommender systems}

Different types of recommender systems serve distinct purposes. This article will focus on content-based methods, hybrid methods and primarly on to collaborative filtering methods.\ref{fig:fig2}
% bellow i am going to describe pros and cons of all the methods and try to find the most optimal one

\begin{figure}[H]
    \includegraphics[width=1\textwidth]{./diagrams/recommender_systems_types.png}
    \caption{Types of recommender systems referenced in this article. Adapted from \cite{10113923}}
    \label{fig:fig2}
\end{figure}

\section{Trigger and triggered model}
Trigger and triggered model (TT) is an enhanced version of a recommendation system which provides anonymous recommendations, when user prioritizes privacy.

\section{Recommendation algorithm}

% \section{Results}

% \section{Discussion}

% \section{Conclusion}


%\acknowledgement{Ak niekomu chcete poďakovať\ldots}


% týmto sa generuje zoznam literatúry z obsahu súboru literatura.bib podľa toho, na čo sa v článku odkazujete

\bibliographystyle{plain} % prípadne alpha, abbrv alebo hociktorý iný
\bibliography{literature}
\end{document}